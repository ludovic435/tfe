% ************************** Thesis Abstract *****************************
% Use `abstract' as an option in the document class to print only the titlepage and the abstract.
\begin{abstract}

Ces dernières années, l'étanchéité à l'air d'un bâtiment prend une importance de plus en plus grande dans l'établissement du bilan énergétique. 

La mise en œuvre d'isolations thermiques toujours plus importantes a pour conséquence d'augmenter proportionnellement l'impact des pertes par in-exfiltration sur le bilan énergétique global du bâtiment. Outre une économie d'énergie, une bonne étanchéité à l'air peut contribuer à éviter des problèmes d'acoustique, de condensation et d'entrée d'odeurs déplaisantes. 

Aujourd'hui, le débit de fuite d'un bâtiment est fixé par les auteurs de projet afin d'atteindre la performance énergétique visée. Toutefois, seul le test d'infiltrométrie qui intervient à la fin du chantier permet d'établir le débit de fuite global du bâtiment. 

Ce travail développe une méthode qui permet de calculer le débit de fuite d'un bâtiment dès la phase de conception. Celle-ci utilise des bases de données reprenant différents matériaux de construction. Cette méthode permet ainsi aux auteurs de projets et à l'entrepreneur de faire le bon choix en matière de matériaux et de techniques en vue d'obtenir un débit de fuite conforme au bilan énergétique. 



\end{abstract}
